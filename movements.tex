\achapter{Movements}{chap:movements}

\def\numfrets{6}

\section{The One}

The one major seven can be a really boring chord to stay on, so it helps to give it some movement.

Here's a Barry Harris-esque way to do it, stolen from the string backgrounds of Peggy Lee's recording of Sing A Rainbow.

\chords{
\abschord{4}{x,8,7,9,8,x}{{Fmaj9}}
\abschord{4}{x,8,7,7,6,x}{{F6}}
\abschord{4}{x,8,7,6,5,x}{{Fmaj7$\sharp$5}}
\abschord{4}{x,8,7,7,6,x}{{F6}}
}

Here's how Burt Bacharach's `Alfie' does it:

\chords{
\abschord{3}{6,x,5,6,8,8}{{A7($\sharp$9)/B$\flat$}}
\abschord{3}{6,x,5,5,6,5}{{B$\flat$maj13}}
}

\section{Minor ii-V}

These have the same top note as the first two chords of "Old Folks".

\chords{
\abschord{2}{x,7,5,3,3,5}{{Em11($\flat$5)}}
\abschord{2}{5,x,5,3,4,5}{{A7($\flat$9)($\sharp$11)}}
}

And another way to do it:

\chords{
\abschord{3}{x,7,8,7,5,5}{{Em11($\flat$5)}}
\abschord{3}{5,x,8,6,5,5}{{A7($\flat$9)}}
}

And here's one using the bebop ``move it up a minor third to make the ii-V" strategy:

\chords{
\abschord{3}{x,7,8,7,5,5}{{Em11($\flat$5)}}
\abschord{6}{x,10,11,10,10,8}{{A7($\sharp$5)($\sharp$9)}}
}

\section{One Note Moves}

This section is named as such because we start with a basic skeleton of the $\flat$7 on the 4th string, the 9th on the 3rd string, and the root on the 1st string.  By moving the note on the 2nd string by half steps, we can generate a bunch of interesting and useful voicings.  Note that the root may be omitted on any of these, or you can grab it with your thumb.

We'll start out with a Dm9 voicing with a half step rub between the 9 and the $\flat$3:

\chords{
\abschord{5}{10,x,10,9,6,10}{{Dm9}}
}

Move the b3 to the 3, and we end up with a nice consonant D9 voicing:

\chords{
\abschord{5}{10,x,10,9,7,10}{{D9}}
}

Move the 3 to the 4, and we end up with a D9sus4 voicing.  I call this chord color the ``gospel V" -- you might also call it C/D or Am7/D.

\chords{
\abschord{5}{10,x,10,9,8,10}{{Dm9sus4}}
}

With the previous two voicings plus one more one-note move, you can also do a nice ii-V.  Note that the easiest fingering for the third chord is to catch both the 4th and 3rd strings with your middle finger.

\chords{
\abschord{5}{x,x,10,9,8,10}{{Am11}}
\abschord{5}{x,x,10,9,7,10}{{D9}}
\abschord{5}{x,x,9,9,7,10}{{Gmaj13}}
}

Move the 4 to the $\sharp$11, and we end up with a nice D9$\sharp$11 voicing.  The third on the bottom is optional.

\chords{
\abschord{6}{x,9,10,9,9,10}{{D9$\sharp$11}}
}

Move the $\sharp$11 to the 5, and we end up with the basic D9 voicing that we all know so well.

\chords{
\abschord{6}{x,9,10,9,10,10}{{D9}}
}

Note that we can move the bottom note from the 3 to the 4 to get a D9sus4 that we can think of as a rootless Am11 to make a ii-V:

\chords{
\abschord{6}{x,10,10,9,10,10}{{Am11}}
\abschord{6}{x,9,10,9,10,10}{{D9}}
}

Move the 5 to the $\sharp$5, and we end up with a D+9.

\chords{
\abschord{7}{x,9,10,9,11,10}{{D+9}}
}

Move the $\sharp$5 to the 13, and we end up with a nice D13 voicing.

\chords{
\abschord{7}{x,9,10,9,12,10}{{D13}}
}

Again, you can move the bottom note from the 3 to the 4 to get an Am11 sound to make a ii-V:

\chords{
\abschord{6}{x,10,10,9,12,10}{{Am11}}
\abschord{6}{x,9,10,9,12,10}{{D13}}
}

\section{Clare Fischer}

There's a reason Herbie Hancock cites this guy as an influence.

\subsection{Elizete with Cal Tjader}

Here's a minor ii-V in A minor.  The ear buys the Emaj7$\sharp$9 as a bitonal hybrid of E major (the V chord) and B7$\sharp$5 (the II-7 chord).

\chords{
\def\numfrets{6}
\abschord{5}{7,8,7,7,8,x}{{Bm13$\flat$5}}
\abschord{6}{o,11,9,8,8,x}{{Emaj7($\sharp$9)}}
\abschord{3}{o,5,6,5,6,x}{{E7($\sharp$5)($\flat$9)}}
\abschord{3}{5,x,9,5,5,x}{{Am(add2)}}
}

And a major ii-V in C major (with an extra passing chord at the end):

\chords{
\def\numfrets{6}
\abschord{3}{x,8,9,5,5,x}{{Dm13}}
\abschord{3}{x,8,9,6,5,x}{{G13($\sharp$11)}}
\chord{t}{x,p{3},p{2},p{4},p{3},x}{{Cmaj13}}
\chord{t}{p{2},x,p{2},p{3},p{1},x}{{G$\flat$7($\sharp$11)}}
}

\subsection{Pensativa}

This example is transcribed/adapted from Clare's comping on the A section of the head.

\chords{
\abschord{6}{x,9,8,8,9,9}{{G$\flat$6/9}}
\abschord{6}{11,10,9,9,10,9}{{E$\flat$7($\sharp$5)($\flat$9)($\sharp$11)}}
\abschord{6}{10,x,9,11,10,9}{{Dmaj13}}
\abschord{6}{x,x,12,11,9,8}{{A$\flat$7($\flat$5)}}
}

Comping behind Bud's solo, Clare plays this voicing for the second chord in the sequence instead:
\chords{
\abschord{6}{x,10,11,9,10,9}{{E$\flat$7($\flat$9)($\sharp$11)}}
}

\subsection{Morning}

I have a YouTube video lesson on these sequences.

\subsubsection{Verse}

In the first sequence, note the contrapuntal motion.  The soprano voice moves up chromatically while first the tenor voice and then the bass voice moves down chromatically.

With voicings with more than one note on the string, the inner voice moves down.

\chords{
\def\numfrets{7}
\chord{6}{p{2},p{3},p{3},p{2},p{6},x}{{Cm7($\flat$5)}}
\chord{6}{p{2},p{3},p{2},p{2},p{7},x}{{Cm7($\flat$5)}}
\chord{6}{x,p{2},p{1},p{2},p{3},p{3}}{{F7($\sharp$5)($\sharp$9)}}
\chord{2}{x,x,p{5},p{4},p{4},p{3}}{{B$\flat$mi/maj7}}
\chord{2}{x,x,p{4},p{4},p{4},p{4}}{{B$\flat$m7}}
\chord{2}{x,x,p{3},p{4},p{4},p{6}}{{E$\flat$13}}
}

\chords{
\def\numfrets{7}
\chord{6}{p{2},p{3},p{3},p{2},p{6},x}{{Cm7($\flat$5)}}
\chord{6}{p{2},p{3},p{2},p{2},p{7},x}{{Cm7($\flat$5)}}
\chord{6}{x,p{2},p{1},p{2},p{3},p{3}}{{F7($\sharp$5)($\sharp$9)}}
\chord{2}{p{4},x,p{4},p{4},p{4},p{6}}{{B$\flat$m9}}
\chord{2}{x,p{4},p{3},p{4},p{4},p{3}}{{E$\flat$9($\sharp$11)}}
}

\chords{
\chord{8}{p{3},x,p{3},p{3},p{3},x}{{E$\flat$m7}}
\chord{8}{x,p{3},p{2},p{2},p{2},x}{{A$\flat$13($\flat$9)}}
\chord{5}{p{4},n,p{3},p{3},p{3}p{4},p{3}}{{D$\flat$maj13($\sharp$11)}}
\chord{5}{x,p{2},p{3},p{3},p{2},p{3}}{{G$\flat$13($\sharp$11)}}
}

\chords{
\chord{5}{p{3},x,p{3},p{4},p{4}p{5},p{6}}{{C13($\sharp$9)}}
\chord{5}{x,p{3},p{2},p{3},p{2}p{4},p{4}}{{F7($\sharp$5)($\flat$9)($\sharp$9)}}
\chord{7}{p{2},x,p{2},p{2},p{2},p{2}}{{Cm7}}
}

\subsection{Samba da Borboleta}

This is a cool tune on Bossa Nova Jazz Samba, almost a contrafact of Take The A Train.

This excerpt is the second A section, coming into the bridge.

\chords{
\chord{6}{p{3},p{2},p{2},p{2},p{3},p{2}}{{Cmaj13}}
\chord{8}{p{3},p{2},p{3},p{2},p{2},p{3}}{{D9($\sharp$11)}}
\chord{7}{x,p{2},p{4},p{3},p{2},p{1}}{{Dm11/13}}
\chord{7}{x,p{2},p{3},p{2},p{3},p{4}}{{G+7($\flat$9)}}
\chord{6}{p{3},p{2},p{2},p{2},p{3},p{3}}{{C6/9}}
\chord{5}{p{4},x,p{4},p{2},p{3},p{1}}{{C13($\flat$9)($\sharp$11)}}
}