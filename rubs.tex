\achapter{Rubs}{chap:rubs}

These refer to chords with either a half-step or whole-step rub in them.  On guitar voicings, often this rub ends up between the 2nd and 3rd strings, since these are a major third apart rather than a fourth.

\def\numfrets{6}

Here's a minor-seventh voicing we've seen earlier with a rub between the 2 and the $\flat$3.  You can omit the bottom note or grab it with your thumb.  Credit goes to Brian Pardo for showing me this one.

\chords{
\abschord{5}{10,x,10,9,6,10}{{Dm9}}
}

Here's a rootless 13th voicing that a piano player might play with a half-step rub in the middle.  Donald Fagen loves this one.  You'll want to catch both of the bottom two notes with your pinky.  Of course, you can omit the top note if desired.  It also makes a nice Am6/9 if you add a sixth-string A.

\chords{
\abschord{4}{x,9,9,5,5,5}{{D13}}
}

And a similar voicing with a different note in the bass gets you a sus($\flat$9) sound.  Credit goes to Damian Garcia for showing me this one.
\chords{
\abschord{3}{6,x,8,4,4,4}{{B$\flat$7sus($\flat$9)}}
}

Here's a nice major seventh voicing with a rub between the M7 (3rd string) and the root (2nd string).  Credit goes to Brian Pardo for showing me this one.  You can omit the bottom note or grab it with your thumb.

\chords{
\abschord{5}{x,8,7,9,6,8}{{Fmaj7}}
}

And a permutation of it with the fifth on the bottom.  I stole this one from Vic Juris.

\chords{
\abschord{5}{8,x,7,9,6,x}{{Fmaj7}}
}

Here's a 7$\sharp$9 voicing with the $\flat$7 on top and a rub between the $\sharp$9 and the 3.

\chords{
\abschord{4}{x,x,5,8,5,6}{{C7$\sharp$9}}
}

And the same idea, but with the root on top and the $\flat$7 on bottom.

\chords{
\abschord{4}{x,x,8,8,5,8}{{C7$\sharp$9}}
}

Here's a dominant voicing with the rub between the 6 and the $\flat$7.  This one is a nice move from the previous chord, from the I to the IV on a blues, for example.

\chords{
\abschord{3}{x,x,7,7,4,8}{{F13}}
}