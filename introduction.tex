\achapter{Introduction}{chap:introduction}

\section{Motivation}

There are a million chord books you can read, and there are several of them that you should probably read before you ever bother with this one:

\begin{itemize}
	\item Joe Pass Guitar Chords
	\item The Barry Harris Harmonic Method for Guitar, by Alan Kingstone
\end{itemize}

And then there are tons of others to pore through: the various Ted Greene tomes (although I find his arrangement sheets much more useful), one by Chuck Wayne, and the Mickey Baker book, just to name a few.

Why, then, am I writing this?  In large part, for my own edification.  If I write something down, it sticks better in my mind, and I can always go back and reference it later if I forget it.

But if I had to state a theme behind this book, it's this: There are chord voicings that don't require insane stretches or place your fingers in a Gordian knot, but still come closer to capturing the full nature of jazz harmony as a pianist or a big band might play it than the standard guitar voicings do.  When I find those voicings, I put them in here.  You may already know all of them -- if so, great!  If not, stick around -- you might learn something.  I know I have.

\section{Guidelines}

To make things cleaner, if a string does not have a dot on it, it is not played unless there is an `o' at the top of the fretboard diagram indicating an open string.

Eventually, I will use Ted Greene's symbols for note movement within a chord.